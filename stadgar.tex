\documentclass[11pt, a4paper]{article}
\usepackage[utf8]{inputenc}
\usepackage[swedish]{babel}
\usepackage{verbatim}
\usepackage{amsmath}
\usepackage{color, xcolor, enumerate}
\usepackage[noend, linesnumbered]{algorithm2e}
\usepackage{graphicx}
\usepackage{caption}
\usepackage{subcaption}

\bibliographystyle{my-style}

\title{Stadgar}
\author{Python Sverige}
\date{Antagits på årsmöte 2013-01-08}

\begin{document}
\maketitle


\section{Namn}
Föreningens namn är Python Sverige.

\section{Ändamål}
Föreningen, som är en ideell förening, partipolitiskt och religiöst obunden, har till uppgift att driva PyCon Sverige, en utvecklarkonferens inriktad på programmeringsspråket Python.

Dessutom kan föreningen enligt årsmötets beslut verka för att inkludera Python i andra konferenser i Skandinavien, i övrigt verka för att främja användandet av programmeringsspråket Python, samt stödja annan relaterad verksamhet.

\section{Hemvist}
Föreningens skall ha sin hemvist i Stockholms kommun, Stockholms län.

\section{Inträde}
Medlemskap i föreningen kan ansökas av enskild person och företag som vill främja föreningens verksamhet och syfte.

\section{Avgifter och medlemskap}
Medlem är den som ansöker om medlemskap till föreningen. Ett företag som är medlem räknas som en röst. Medlemsavgiften är 0 kr.

\section{Uteslutning}
Bryter medlem mot dessa stadgar, skadar på annat sätt föreningen eller motarbetar dess syften kan styrelsen, om den är enhällig, eller årsmötet utesluta medlemmen ur föreningen.

Den som inte godtar styrelsens beslut om uteslutning, äger hänskjuta frågan till årsmötets prövning.

\section{Organ}
Föreningens organ

\begin{itemize}
    \item Årsmöte
    \item Styrelse
    \item Revisorer
\end{itemize}

\section{Årsmöte}
Föreningens årsmöte hålls årligen före den 1 juli. Vid årsmötet har varje medlem en röst. Såväl val som övriga frågor avgörs genom öppen omröstning om inte sluten begärs. Votering vid personval skall ske med slutna sedlar. Vid lika röstetal har årsmötets ordförande utslagsröst. Vid personval avgör dock lotten.

Vid ordinarie årsmöte skall följande ärenden behandlas:

\begin{enumerate}
    \item Val av mötesordförande och sekreterare
    \item Val av två personer att jämte mötesordföranden justera årsmötets protokoll
    \item Fastställande av föredragningslistan
    \item Fastställande av röstlängd
    \item Godkännande av kallelse
    \item Styrelsens verksamhetsberättelse och ekonomiska berättelse över det senaste året
    \item Revisorernas berättelse
    \item Fråga om fastställande av balansräkning samt disposition av årets resultat
    \item Fråga om ansvarsfrihet för styrelsens ledamöter
    \item Beslut om:
    \begin{enumerate}
        \item antal ledamöter i styrelsen och ev. antal ersättare
        \item mandatperiod
        \item ev ersättning till styrelse m fl.
    \end{enumerate}
    \item Val av ordförande i föreningen
    \item Val av övriga styrelseledamöter samt ev. ersättare
    \item Val av revisorer och ersättare
    \item Val av valberedning
    \item Beslut om eventuella regler och belopp för nästkommande års medlemsavgift
    \item Framställningar och förslag från styrelsen och från medlemmar som inkommit till styrelsen senast 10 dagar före årsmötet
    \item Vid årsmötet väckta frågor
\end{enumerate}

\section{Extra årsmöte}
Extra årsmöte hålls, när styrelsen eller revisorerna finner att det är nödvändigt eller när minst 1/10 av föreningens medlemmar så kräver genom skriftlig begäran till styrelsen.

Av begäran skall framgå det eller de ärenden som medlemmarna vill ha behandlat. På extra årsmöte får endast behandlas de ärenden som angivits i kallelse.

\section{Kallelse till årsmöte}
Kallelse till årsmöte skall ske genom kallelse postad till föreningens email-lista senast 30 dagar före stämman.

\section{Styrelse}
Föreningens angelägenheter sköts av en styrelse bestående av minst 3 ledamöter och det antal ersättare som årsmötet beslutar.

Mandatperioden kan vara på ett eller två år, enligt årsmötets beslut.

\subsection{}
Styrelsen sammanträder när ordföranden finner det lämpligt eller då minst två av styrelsens ledamöter hos ordföranden skriftligen begär sammanträde.

\subsection{}
Styrelsen är beslutsför, när de närvarandes antal överstiger hälften av hela antalet ledamöter.

Beslut fattas med enkel majoritet. Vid lika röstetal har ordföranden utslagsröst utom vid personval då lotten avgör.

\subsection{}
Styrelsen skall i enlighet med dessa stadgar sköta föreningens angelägenheter.

Styrelsen skall föra en medlemsförteckning. Av förteckningen skall framgå medlemmarnas fullständiga namn och adress.

\section{Teckningsrätt}
Styrelsen i sin helhet eller de som styrelsen utser har föreningens teckningsrätt.

\section{Räkenskaper}
Föreningens räkenskaper omfattar tiden från den 1 januari till den 31 december.

Styrelsen skall senast 14 dagar före ordinarie årsmöte överlämna sina redovisningshandlingar
till revisorerna.

\section{Revisorer}
Föreningens räkenskaper och styrelsens förvaltning skall granskas av minst revisor av ordinarie årsmöte utsedd revisor.

För revisorerna skall utses ersättare. Revisorer och ersättare för dessa utses för ett år. Revisionsberättelse skall lämnas till styrelsen senast 7 dagar före ordinarie årsmötet.

\section{Ändring av föreningens stadgar samt likvidation}
Beslut om ändring av dessa stadgar och om föreningens trädande i likvidation fattas på två på varandra följande årsmöten, varav minst en skall vara ordinarie årsmöte. Beslutet skall för att vara gällande på den senare årsmötet ha biträtts av minst 2/3 av de röstande.

\section{Upplösning}
Skulle föreningen upplösas skall föreningens behållna tillgångar tillfalla Python Software Foundation http://www.python.org/



\end{document}
